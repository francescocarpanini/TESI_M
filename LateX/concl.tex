\chapter*{Conclusion}
\addcontentsline{toc}{chapter}{\numberline{}Conclusion}
\markboth{Conclusion}{Conclusion}

In conclusion, the results achieved met the initial expectations, successfully delivering the performance required to accomplish the primary objective: enabling the offline localization on HD maps without the need of any manual adjustments.

As outlined in dedicated sections, there are numerous potential opportunities for further development. These include maintaining the current architecture while incorporating more generalized data during the training phase by increasing dimension of the dataset by integrating different scenarios or exploring alternative architectures to improve precision. 

However, following the latter approach would likely increase the computational resources required, making it usable just in offline pipelines, where inference times are not that important.
This work demonstrates a step toward automating localization processes and provides a basis for future improvements aimed at refining performance and expanding applicability.
