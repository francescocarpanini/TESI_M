\chapter*{Conclusion}
\addcontentsline{toc}{chapter}{\numberline{}Conclusion}
\markboth{Conclusion}{Conclusion}

This thesis studied the use of HD maps in autonomous driving. It showed that HD maps are crucial in the field, but their high maintenance cost often makes them impractical. Modern methods use neural networks to create HD maps in real-time, addressing this problem. However, these networks require large datasets to perform well. Therefore, being able to localize the vehicle within the HD map offline becomes essential for creating these datasets.

The thesis reviewed existing methods for HD map-based localization by analyzing the current state-of-the-art. This review demonstrated how these methods assist in generating datasets but revealed that achieving the required precision still requires human refinement.

By examining the inner workings of neural networks and customizing architectures, this research goes beyond traditional fine-tuning and provides valuable insights into developing more robust and adaptable models.

Two approaches were explored in this research.
The first approach used features extracted from detectors as inputs, combined with relevant sections of the HD map.
The second approach used captured images to reconstruct a bird's-eye view (BEV), which was then integrated with the network inputs.

The results of the various approaches and architectures are discussed in detail within this work. Notably, the second approach achieved the research goal of enabling offline localization on HD maps without requiring manual adjustments.

As highlighted in the final Chapter, there are many opportunities for future development. These include improving the current architecture by training with more generalized data, expanding the dataset to cover a broader range of scenarios, or exploring alternative architectures to enhance precision.

This research represents a step forward in automating localization processes for HD maps. By reducing the need for manual adjustments and simplifying complex workflows, it provides a cost-effective and scalable solution for creating ground truth datasets. This approach supports real-time HD map generation networks, making them more practical and efficient.