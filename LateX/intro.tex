\NoBgThispage
\chapter*{Introduction}
\addcontentsline{toc}{chapter}{\numberline{}Introduction}
\markboth{Introduction}{Introduction}

This thesis presents MapAlign, a neural network specifically designed to automatically generate ground truth data. This ground truth is used to create a dataset for training a deeper neural network capable of generating HD maps in real time. As expected, such a network requires a robust and comprehensive dataset to ensure effective training and reliable performance.

The thesis is structured as follows:

Chapter 1 introduces the concepts of autonomous driving, the role of maps, and the importance of HD maps. Additionally, it provides an overview of the state-of-the-art localization techniques for HD maps.

Chapter 2 offers an introduction to neural networks and describes the dataset necessary to train MapAlign. It highlights the types of data required and the preparation steps involved.

Chapter 3 explores the various neural network architectures, presenting the results obtained from different attempts and configurations.

Chapter 4 concludes with a comparison of the results to identify the best trade-off between performance and efficiency. Future developments and potential improvements are also discussed, focusing on enabling real-world applications of the proposed model.

This work aims to contribute to the field of autonomous driving by simplifying the generation of HD map datasets, ultimately facilitating the development of more advanced and reliable mapping systems.